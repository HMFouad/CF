\documentclass[a4paper]{book}
\usepackage{fullpage}

\usepackage[utf8]{inputenc}
\usepackage[T1]{fontenc}
\usepackage[francais]{babel}

\usepackage{latexsym}
\usepackage{fancyhdr}
\usepackage{makeidx}
\usepackage{graphics}
\usepackage{graphicx}
\usepackage{longtable}
\usepackage{moreverb}
\usepackage{listings}

\newcommand{\altarica}{{\sc AltaRica}}

\begin{document}

\title{Master 1, Conceptions Formelles\\
Projet du module \altarica\\
Synthèse (assistée) d'un contrôleur du niveau d'une cuve}

\date{}

\author{HOCINI Mohamed Fouad \and Ismael Traore \and Fairouz Elouazi}

\maketitle

\chapter{Le sujet}
\section{Cahier des charges}

Le système que l'on souhaite concevoir est composé~:
\begin{itemize}
\item d'un réservoir contenant {\bf toujours} suffisamment d'eau pour alimenter l'exploitation,
\item d'une cuve,
\item de deux canalisations parfaites amont reliant le réservoir à la cuve, et permettant d'amener l'eau à la cuve,
\item d'une canalisation parfaite aval permettant de vider l'eau de la cuve,
\item chaque canalisation est équipée d'une vanne commandable, afin de réguler l'alimentation et la vidange de la cuve,
\item d'un contrôleur.
\end{itemize}

\subsection{Détails techniques}

\subsubsection{La vanne}
Les vannes sont toutes de même type, elles possèdent trois niveaux de débits correspondant à trois diamètres d'ouverture~: 0 correspond à la vanne fermée, 1 au diamètre intermédiaire et 2 à la vanne complètement ouverte. Les vannes sont commandables par les deux instructions {\tt inc} et {\tt dec} qui respectivement augmente et diminue l'ouverture. Malheureusement, la vanne est sujet à défaillance sur sollicitation, auquel cas le système de commande devient inopérant, la vanne est désormais pour toujours avec la même ouverture.

\subsubsection{La Cuve}
Elle est munie de $nbSensors$ capteurs (au moins quatre) situés à $nbSensors$ hauteurs qui permettent de délimiter $nbSensors+1$ zones. La zone 0 est comprise entre le niveau 0 et le niveau du capteur le plus bas; la zone 1 est comprise entre ce premier capteur et le second, et ainsi de suite.

Elle possède en amont un orifice pour la remplir limité à un débit de 4, et en aval un orifice pour la vider limité à un débit de 2.  

\subsubsection{Le contrôleur}
Il commande les vannes avec les objectifs suivants ordonnés par importance~:
\begin{enumerate}
\item Le système ne doit pas se bloquer, et le niveau de la cuve ne doit jamais atteindre les zones 0 ou $nbSensors$.
\item Le débit de la vanne aval doit être le plus important possible.
\end{enumerate}

On fera également l'hypothèse que les commandes ne prennent pas de temps, et qu'entre deux pannes et/ou cycle {\em temporel}, le contrôleur à toujours le temps de donner au moins un ordre. Réciproquement, on fera l'hypothèse que le système à toujours le temps de réagir entre deux commandes.

\subsubsection{Les débits}
Les règles suivantes résument l'évolution du niveau de l'eau dans la cuve~:
\begin{itemize}
\item Si $(amont > aval)$ alors au temps suivant, le niveau aura augmenté d'une unité.
\item Si $(amont < aval)$ alors au temps suivant, le niveau aura baissé d'une unité.
\item Si $(amont = aval = 0)$ alors au temps suivant, le niveau n'aura pas changé.
\item Si $(amont = aval > 0)$ alors au temps suivant, le niveau pourra~:
  \begin{itemize}
  \item avoir augmenté d'une unité,
  \item avoir baissé d'une unité,
  \item être resté le même.
  \end{itemize}
\end{itemize}

\section{L'étude}

\subsection{Rappel méthodologique}
Comme indiqué en cours, le calcul par point fixe du contrôleur est exact, mais l'opération de projection effectuée ensuite peut perdre de l'information et générer un contrôleur qui n'est pas satisfaisant. Plus précisemment, le contrôleur \altarica\ généré~:
\begin{itemize}
\item ne garanti pas la non accessibilité des \emph{Situations Redoutées}.
\item ne garanti pas l'absence de \emph{nouvelles situations de blocages}.
\end{itemize}

Dans le cas ou il existe toujours \emph{des situations de blocages ou redoutées}, vous pouvez au choix~:
\begin{enumerate}
\item Corriger manuellement le contrôleur calculé (sans doute très difficile).
\item Itérer le processus du calcul du contrôleur jusqu'à stabilisation du résultat obtenu. 
  \begin{itemize}
  \item Si le contrôleur obtenu est sans blocage et sans situation redoutée, il est alors correct.
  \item Si le contrôleur obtenu contient toujours des blocages ou des situations redoutées, c'est que le contrôleur initial n'est pas assez performant, mais rien de garanti que l'on soit capable de fournir ce premier contrôleur suffisemment performant.
  \end{itemize}
\end{enumerate}

{\bf Remarque} : Pour vos calculs, vous pouvez utiliser au choix les commandes~:
\begin{itemize}
\item {\tt altarica-studio xxx.alt xxx.spe}
\item {\tt arc -b xxx.alt xxx.spe}
\item {\tt make} pour utiliser le fichier GNUmakefile fourni.
\end{itemize}

\subsection{Le travail a réaliser}

Avant de calculer les contrôleurs, vous devez répondre aux questions suivantes.
\begin{enumerate}
\item Expliquez le rôle de la constante $nbFailures$ et de la contrainte, présente dans le composant {\tt System}, $nbFailures >= (V[0].fail + V[1].fail + V[2].fail)$.
\item Expliquez le rôle du composant {\tt ValveVirtual} et de son utilisation dans le composant {\tt CtrlVV}, afin de remplacer le composant {\tt Ctrl} utilisé en travaux dirigés.
\end{enumerate}

L'étude consiste à étudier le système suivant deux paramètres~:
\begin{enumerate}
\item $nbFailures$~: une constante qui est une borne pour le nombre de vannes pouvant tomber en panne.
\item Le contrôleur initial qui peut être soit {\tt Ctrl}, soit {\tt CtrlVV}.
\end{enumerate}

Pour chacun des huit systèmes étudiés, vous devez décrire votre méthodologie pour calculer les différents contrôleurs et répondre aux questions suivantes~:

\begin{enumerate}
\item Est-il possible de contrôler en évitant les blocages et les situations critiques ?
\item Si oui, donnez quelques caractéristiques de ce contrôleur, si non, expliquez pourquoi.
\item Est-il possible de contrôler en optimisant le débit aval et en évitant les blocages et les situations critiques ?
\item Si oui, donnez quelques caractéristiques de ce contrôleur, si non, expliquez pourquoi.
\end{enumerate}


\chapter{Le rapport}
\section{Rôle de la constante {\tt nbFailures} (2 points)}
la constante nbFaillures sert à borner le nombre total de vanne pouvant tomber en panne ,la contrainte permet d'imposer les valeurs de la constante nbFailures dans l'intervalle 
[0,3[.
\section{Résultats avec le contrôleur initial {\tt Ctrl}}

\subsection{Calcul d'un contrôleur}

\subsubsection{Avec 0 défaillance (1 point)}
\lstinputlisting{Res/System0FCtrl.res}
\lstinputlisting{Res/System0FCtrl0F1I.res}
\lstinputlisting{Res/System0FCtrl0F2I.res}
\lstinputlisting{Res/System0FCtrl0F3I.res}
\lstinputlisting{Res/System0FCtrl0F4I.res}
\paragraph{Interprétation des résultats}
Dans le cas où les vannes ne sont pas défaillantes, on constate qu'il est possible d'éviter que le système se bloque, ainsi que les situations critiques.

On peut donc déduire qu'une des caractéristique de ce contrôleur est qu'il doit gérer les pannes.

Suite à l'analyse du débit de la vanne de sortie avec les différentes variables out0, out1, out2 on remarque que le contrôleur ne contrôle pas de façon optimale le débit aval.

D'après les résultats de ces variables, le contrôleur peut ouvrir partiellement ou totalement la vanne et éventuellement la fermer.
\subsubsection{Avec 1 défaillance (1 point)}
\lstinputlisting{Res/System1FCtrl.res}
\lstinputlisting{Res/System1FCtrl1F1I.res}
\lstinputlisting{Res/System1FCtrl1F2I.res}
\lstinputlisting{Res/System1FCtrl1F3I.res}
\lstinputlisting{Res/System1FCtrl1F4I.res}
\paragraph{Interprétation des résultats}
D'après les résultats, il n'est pas possible de détecter que le système soit bloquer et les situations critiques pouvant survenir au cours du fonctionnement du système.

Cette situation est dû au faite que le contrôleur initial ne gère pas le cas où le système est défaillant.

Il n'est pas possible de contrôler en optimisant le débit aval, car la condition préalable d'éviter les blocages et les situations critiques n'est pas satisfaite.

\subsubsection{Avec 2 défaillances (1 point)}
\lstinputlisting{Res/System2FCtrl.res}
\lstinputlisting{Res/System2FCtrl2F1I.res}
\lstinputlisting{Res/System2FCtrl2F2I.res}
\lstinputlisting{Res/System2FCtrl2F3I.res}
\lstinputlisting{Res/System2FCtrl2F4I.res}
\paragraph{Interprétation des résultats}
Comme dans le cas précédent où le système a une défaillance, on observe aussi l'apparition de blocages et de situations critiques, donc les remarques précédentes s'appliquent aussi dans ce cas.

\subsubsection{Avec 3 défaillances (1 point)}
\lstinputlisting{Res/System3FCtrl.res}
\lstinputlisting{Res/System3FCtrl3F1I.res}
\lstinputlisting{Res/System3FCtrl3F2I.res}
\lstinputlisting{Res/System3FCtrl3F3I.res}
\lstinputlisting{Res/System3FCtrl3F4I.res}
\paragraph{Interprétation des résultats}
Comme dans les deux cas précédents où le système a une ou deux défaillance(s), on observe aussi l'apparition de blocages et de situations critiques, donc les remarques précédentes s'appliquent aussi dans ce cas.

\subsection{Calcul des contrôleurs optimisés (2 points)}
Il a fallu intégrer la gestion des pannes en observant les pannes survenant lors du fonctionnement du système. Une simple assert dans le composant Système et notre contrôleur est capable de détecter les pannes.

\subsubsection{Avec 0 défaillance}
\lstinputlisting{Res/System0FCtrlOpt.res}
\lstinputlisting{Res/System0FCtrlOpt0F1I.res}
\lstinputlisting{Res/System0FCtrlOpt0F2I.res}
\lstinputlisting{Res/System0FCtrlOpt0F3I.res}
\lstinputlisting{Res/System0FCtrlOpt0F4I.res}

\subsubsection{Avec 1 défaillance }
\lstinputlisting{Res/System1FCtrlOpt.res}
\lstinputlisting{Res/System1FCtrlOpt1F1I.res}
\lstinputlisting{Res/System1FCtrlOpt1F2I.res}
\lstinputlisting{Res/System1FCtrlOpt1F3I.res}
\lstinputlisting{Res/System1FCtrlOpt1F4I.res}

\subsubsection{Avec 2 défaillances }
\lstinputlisting{Res/System2FCtrlOpt.res}
\lstinputlisting{Res/System2FCtrlOpt2F1I.res}
\lstinputlisting{Res/System2FCtrlOpt2F2I.res}
\lstinputlisting{Res/System2FCtrlOpt2F3I.res}
\lstinputlisting{Res/System2FCtrlOpt2F4I.res}

\subsubsection{Avec 3 défaillances }
\lstinputlisting{Res/System3FCtrlOpt.res}
\lstinputlisting{Res/System3FCtrlOpt3F1I.res}
\lstinputlisting{Res/System3FCtrlOpt3F2I.res}
\lstinputlisting{Res/System3FCtrlOpt3F3I.res}
\lstinputlisting{Res/System3FCtrlOpt3F4I.res}

\section{Rôle des composants {\tt ValveVirtual} et {\tt CtrlVV} (4 points)}
Le composant ValveVirtual comme son nom l'indique sert à simuler le comportement d'une vanne en parfait état et stopper son utilisation lorsque cette dernière est défaillante.
L'utilisation des ValveVirtual au sein du CtrlVV permet de gérer les pannes pouvant intervenir lors de l'utilisation d'une vanne afin d'éviter des situations de blocages et critiques (atteinte de la zone 0 ou de la zone nbSensors). Il est préférable de l'utiliser par rapport au contrôleur initiale car le comportement du contrôleur est aléatoire et ne gère pas le cas des pannes.    
   
\section{Résultats avec le contrôleur initial {\tt CtrlVV}}

\subsection{Calcul d'un contrôleur}

\subsubsection{Avec 0 défaillance (1 point)}
\lstinputlisting{Res/System0FCtrlVV.res}
\lstinputlisting{Res/System0FCtrlVV0F1I.res}
\lstinputlisting{Res/System0FCtrlVV0F2I.res}
\lstinputlisting{Res/System0FCtrlVV0F3I.res}
\lstinputlisting{Res/System0FCtrlVV0F4I.res}
\paragraph{Interprétation des résultats}
Dans le cas où les vannes ne sont pas défaillantes, on constate qu'il est possible d'éviter que le système se bloque, ainsi que les situations critiques.

On peut donc déduire qu'une des caractéristique de ce contrôleur est qu'il est capable de gérer les pannes.

Suite à l'analyse du débit de la vanne de sortie avec les différentes variables out0, out1, out2 on remarque que le contrôleur ne contrôle pas de façon optimale le débit aval.

D'après les résultats de ces variables, le contrôleur peut ouvrir partiellement ou totalement la vanne et éventuellement la fermer.
\subsubsection{Avec 1 défaillance (1 point)}
\lstinputlisting{Res/System1FCtrlVV.res}
\lstinputlisting{Res/System1FCtrlVV1F1I.res}
\lstinputlisting{Res/System1FCtrlVV1F2I.res}
\lstinputlisting{Res/System1FCtrlVV1F3I.res}
\lstinputlisting{Res/System1FCtrlVV1F4I.res}
\paragraph{Interprétation des résultats}
D'après les résultats, il est possible de détecter que le système soit bloquer et les situations critiques pouvant survenir au cours du fonctionnement du système.

Cette situation est dû au faite que le contrôleur à vannes simulées gère le cas où le système est défaillant en bloquant la vanne en cas de défaillance.


\subsubsection{Avec 2 défaillances (1 point)}
\lstinputlisting{Res/System2FCtrlVV.res}
\lstinputlisting{Res/System2FCtrlVV2F1I.res}
\lstinputlisting{Res/System2FCtrlVV2F2I.res}
\lstinputlisting{Res/System2FCtrlVV2F3I.res}
\lstinputlisting{Res/System2FCtrlVV2F4I.res}
\paragraph{Interprétation des résultats}
Comme dans le cas précédent où le système a une défaillance, il est possible de détecter l'apparition de blocages et de situations critiques, donc les remarques précédentes s'appliquent aussi dans ce cas.
\subsubsection{Avec 3 défaillances (1 point)}
\lstinputlisting{Res/System3FCtrlVV.res}
\lstinputlisting{Res/System3FCtrlVV3F1I.res}
\lstinputlisting{Res/System3FCtrlVV3F2I.res}
\lstinputlisting{Res/System3FCtrlVV3F3I.res}
\lstinputlisting{Res/System3FCtrlVV3F4I.res}
\paragraph{Interprétation des résultats}
Comme dans les deux cas précédents où le système a une ou deux défaillance(s), il est possible de détecter l'apparition de blocages et de situations critiques, donc les remarques précédentes s'appliquent aussi dans ce cas.
\subsection{Calcul des contrôleurs optimisés }
Pour optimiser ce contrôleur nous avons du intégrer une vanne qui ne tombe jamais en panne pour rendre le système plus robuste au lieu de bloquer la vanne comme le fait le contrôleur précédent. 

\subsubsection{Avec 0 défaillance}
\lstinputlisting{Res/System0FCtrlVR.res}
\lstinputlisting{Res/System0FCtrlVR0F1I.res}
\lstinputlisting{Res/System0FCtrlVR0F2I.res}
\lstinputlisting{Res/System0FCtrlVR0F3I.res}
\lstinputlisting{Res/System0FCtrlVR0F4I.res}

\subsubsection{Avec 1 défaillance }
\lstinputlisting{Res/System1FCtrlVR.res}
\lstinputlisting{Res/System1FCtrlVR1F1I.res}
\lstinputlisting{Res/System1FCtrlVR1F2I.res}
\lstinputlisting{Res/System1FCtrlVR1F3I.res}
\lstinputlisting{Res/System1FCtrlVR1F4I.res}

\subsubsection{Avec 2 défaillances (1 point)}
\lstinputlisting{Res/System2FCtrlVR.res}
\lstinputlisting{Res/System2FCtrlVR2F1I.res}
\lstinputlisting{Res/System2FCtrlVR2F2I.res}
\lstinputlisting{Res/System2FCtrlVR2F3I.res}
\lstinputlisting{Res/System2FCtrlVR2F4I.res}

\subsubsection{Avec 3 défaillances (1 point)}
\lstinputlisting{Res/System3FCtrlVR.res}
\lstinputlisting{Res/System3FCtrlVR3F1I.res}
\lstinputlisting{Res/System3FCtrlVR3F2I.res}
\lstinputlisting{Res/System3FCtrlVR3F3I.res}
\lstinputlisting{Res/System3FCtrlVR3F4I.res}
\section{Conclusion (2 points)}
Notre choix final portera sur le System3FCtrlVR car il permet d'éviter à la fois les situation redoutées tels que le blocage du sytème et les niveaux critiques. De plus il nous permet de maximiser le débit de la vanne aval en utilisant les priorités d'Altarica. Le but est de privilégier l’augmentation du débit de la vanne aval par rapport aux autres actions, et privilégier la stagnation du débit par rapport à sa diminution.
 Une autre alternative aurait été de réparer la vanne mais les contraintes du cahier des charges ne le permettent pas. 
 
\end{document}
